\documentclass[final]{siamltex}

%%%%%%%%%%%%%%%%%%%%%%%%%%%%%%%%%%%%%%%%%%%%%%%%%%
\usepackage{rotfig_tikz}
%%%%%%%%%%%%%%%%%%%%%%%%%%%%%%%%%%%%%%%%%%%%%%%%%%
\usepackage{multicol}
\usepackage{fancyvrb}
\newtheorem{example}[theorem]{\it Example}
% some color definitions
\usepackage{xcolor}


\title{\texttt{rotfig\_tikz} User's Guide}

\author{Thomas Mach\footnotemark[3]}

\begin{document}
\maketitle

\renewcommand{\thefootnote}{\fnsymbol{footnote}}

\footnotetext[3]{%
  Department of Computer Science, KU~Leuven, Celestijnenlaan 200A, 3001 Leuven
  (Heverlee), Belgium;
  \mbox{(\texttt{thomas.mach@cs.kuleuven.be})}.}%

\renewcommand{\thefootnote}{\arabic{footnote}}


\date{\today}
\maketitle

%%%%%%%%%%%%%%%%%%%%%%%%%%%%%%%%%%%%%%%%%%%%%%%%%%%%%%%%%%%%%%%%%%%%%%%%%%%%%%%%
% Abstract
%%%%%%%%%%%%%%%%%%%%%%%%%%%%%%%%%%%%%%%%%%%%%%%%%%%%%%%%%%%%%%%%%%%%%%%%%%%%%%%%
\begin{abstract}
  A brief explanation of \texttt{rotfig\_tikz} for \LaTeX is given. Note that
  \texttt{rotfig\_tikz} is based on Raf Vandebril's Rotational Figures package
  \texttt{rotfig} and does not work without the \texttt{rotfig} package.
\end{abstract}

%%%%%%%%%%%%%%%%%%%%%%%%%%%%%%%%%%%%%%%%%%%%%%%%%%%%%%%%%%%%%%%%%%%%%%%%%%%%%%%%
% Headings
%%%%%%%%%%%%%%%%%%%%%%%%%%%%%%%%%%%%%%%%%%%%%%%%%%%%%%%%%%%%%%%%%%%%%%%%%%%%%%%%
\pagestyle{myheadings} %
\thispagestyle{plain} %
\markboth{T.\ Mach}%
{\texttt{rotfig\_tikz} User's Guide}


%%%%%%%%%%%%%%%%%%%%%%%%%%%%%%%%%%%%%%%%%%%%%%%%%%%%%%%%%%%%%%%%%%%%%%%%%%%%%%%%
% Text
%%%%%%%%%%%%%%%%%%%%%%%%%%%%%%%%%%%%%%%%%%%%%%%%%%%%%%%%%%%%%%%%%%%%%%%%%%%%%%%%
\section{Introduction}
\label{sec:introduction}

The \texttt{rotfig\_tikz} package is a small extension of Raf Vandebril's
\texttt{rotfig} package. While using the \texttt{rotfig} package with TikZ we
noticed that we used certain figures recurrently in our papers. This packages
aims to simplify the usage of these recurring patterns. By always using the
same set of commands we also hope to unify our rotational notation.

\subsection{License}
The MIT License (MIT)

Copyright (c) 2016,
Thomas Mach.

Permission is hereby granted, free of charge, to any person obtaining
a copy of this software and associated documentation files (the
"Software"), to deal in the Software without restriction, including
without limitation the rights to use, copy, modify, merge, publish,
distribute, sublicense, and/or sell copies of the Software, and to
permit persons to whom the Software is furnished to do so, subject to
the following conditions:

The above copyright notice and this permission notice shall be
included in all copies or substantial portions of the Software.

THE SOFTWARE IS PROVIDED "AS IS", WITHOUT WARRANTY OF ANY KIND,
EXPRESS OR IMPLIED, INCLUDING BUT NOT LIMITED TO THE WARRANTIES OF
MERCHANTABILITY, FITNESS FOR A PARTICULAR PURPOSE AND
NONINFRINGEMENT. IN NO EVENT SHALL THE AUTHORS OR COPYRIGHT HOLDERS BE
LIABLE FOR ANY CLAIM, DAMAGES OR OTHER LIABILITY, WHETHER IN AN ACTION
OF CONTRACT, TORT OR OTHERWISE, ARISING FROM, OUT OF OR IN CONNECTION
WITH THE SOFTWARE OR THE USE OR OTHER DEALINGS IN THE SOFTWARE.



\section{Dependencies}
The \texttt{rotfig\_tikz} package is dependent on \texttt{rotfig},
\texttt{xargs}, and \texttt{tikz} including the TikZ libraries \texttt{arrows},
\texttt{calc} and \texttt{decorations.pathreplacing}. All these packages will be
loaded when 
\begin{Verbatim}
\usepackage{rotfig_tikz}
\end{Verbatim}
is used in the preamble of your \LaTeX{} document. 

\newpage{}
\section{Basic operations}
\subsection{tikzrotation}
The most basic operation is to draw a single rotation $\begin{smallmatrix}
  \Rc\\[0.7ex] \rc \end{smallmatrix}$ of standard size in a
\texttt{tikzpicture}.

\begin{multicols}{2}
 \begin{Verbatim}
  \begin{center}
    \begin{tikzpicture}[scale=1.66,y=-1cm]
      \tikzrotation{0.0}{0.2} 
      \node at (0.6,0.2) 
      [align=center] {$\times$};
      \node at (0.6,0.4) 
      [align=center] {$\times$};
    \end{tikzpicture}
  \end{center} 
 \end{Verbatim}
 \columnbreak
  \begin{center}
    \begin{tikzpicture}[scale=1.66,y=-1cm]
      \tikzrotation{0.0}{0.2} 
      \node at (0.6,0.2) 
      [align=center] {$\times$};
      \node at (0.6,0.4) 
      [align=center] {$\times$};
    \end{tikzpicture}
  \end{center} 
\end{multicols}

The first argument is the x-coordinate, the second the y-coordinate of the upper
arrow. If $\times$ are position in the same rows, then the arrows point to the
center of $\times$. \texttt{tikzrotation} has further four optional arguments,
which can be used to mark rotations by color or with a name.

\paragraph{Changing the color}

\begin{multicols}{2}
 \begin{Verbatim}
  \begin{center}
    \begin{tikzpicture}[scale=1.66,y=-1cm]
      \tikzrotation[orange]{0.0}{0.2} 
    \end{tikzpicture}
  \end{center} 
 \end{Verbatim}
 \columnbreak
  \begin{center}
    \begin{tikzpicture}[scale=1.66,y=-1cm]
      \tikzrotation[orange]{0.0}{0.2} 
    \end{tikzpicture}
  \end{center} 
\end{multicols}

\paragraph{Placinc a letter on the right}

\begin{multicols}{2}
 \begin{Verbatim}
  \begin{center}
    \begin{tikzpicture}[scale=1.66,y=-1cm]
      \tikzrotation[black][a]{0.0}{0.2} 
    \end{tikzpicture}
  \end{center} 
 \end{Verbatim}
 \columnbreak
  \begin{center}
    \begin{tikzpicture}[scale=1.66,y=-1cm]
      \tikzrotation[black][a]{0.0}{0.2} 
    \end{tikzpicture}
  \end{center} 
\end{multicols}

\paragraph{Placing a letter below}

\begin{multicols}{2}
 \begin{Verbatim}
  \begin{center}
    \begin{tikzpicture}[scale=1.66,y=-1cm]
      \tikzrotation[black][][b]{0.0}{0.2} 
    \end{tikzpicture}
  \end{center} 
 \end{Verbatim}
 \columnbreak
  \begin{center}
    \begin{tikzpicture}[scale=1.66,y=-1cm]
      \tikzrotation[black][][b]{0.0}{0.2} 
    \end{tikzpicture}
  \end{center} 
\end{multicols}

\paragraph{Placing a letter above}

\begin{multicols}{2}
  \begin{Verbatim}
    \begin{center}
      \begin{tikzpicture}[scale=1.66,y=-1cm]
        \tikzrotation[black][][][c]{0.0}{0.2} 
      \end{tikzpicture}
    \end{center} 
  \end{Verbatim}
  \columnbreak
  \begin{center}
    \begin{tikzpicture}[scale=1.66,y=-1cm]
      \tikzrotation[black][][][c]{0.0}{0.2} 
    \end{tikzpicture}
  \end{center} 
\end{multicols}

\newpage{}
\paragraph{Example}
The QR decomposition of an upper Hessenberg matrix.

\begin{multicols}{2}
  \begin{Verbatim}
  \begin{center}
    \begin{tikzpicture}[scale=1.66,y=-1cm]
      \foreach \j in {0,...,7} { 
        \tikzrotation{\j/5}{\j/5}
      }
      \foreach \j in {0,...,8} { 
        \foreach \i in {\j,...,8} {
          \node at (\i/5+0.6,\j/5)
          [align=center,scale=1.0] {$\times$};
        } 
      }; 
    \end{tikzpicture}
  \end{center} 
  \end{Verbatim}
  \columnbreak
  \begin{center}
    \begin{tikzpicture}[scale=1.66,y=-1cm]
      \foreach \j in {0,...,7} { 
        \tikzrotation{\j/5}{\j/5}
      }
      \foreach \j in {0,...,8} { 
        \foreach \i in {\j,...,8} {
          \node at (\i/5+0.6,\j/5)
          [align=center,scale=1.0] {$\times$};
        } 
      }; 
    \end{tikzpicture}
  \end{center} 
\end{multicols}  


\subsection{tikzrotationsmall}
Sometimes we like to draw small pictures. That is possible with
\texttt{tikzrotationsmall}.  

\begin{multicols}{2}
 \begin{Verbatim}
  \begin{center}
    \begin{tikzpicture}[y=-1cm]
      \tikzrotationsmall{0.0}{0.2} 
    \end{tikzpicture}
  \end{center} 
 \end{Verbatim}
 \columnbreak
  \begin{center}
    \begin{tikzpicture}[y=-1cm]
      \tikzrotationsmall{0.0}{0.2} 
    \end{tikzpicture}
  \end{center} 
\end{multicols}

The first argument is the x-coordinate, the second the
y-coordinate. \texttt{tikzrotationsmall} has the same four optional arguments,
which can be used to mark rotations by color or with a name.

\paragraph{Changing the color}

\begin{multicols}{2}
 \begin{Verbatim}
  \begin{center}
    \begin{tikzpicture}[y=-1cm]
      \tikzrotationsmall[orange]{0.0}{0.2} 
    \end{tikzpicture}
  \end{center} 
 \end{Verbatim}
 \columnbreak
  \begin{center}
    \begin{tikzpicture}[y=-1cm]
      \tikzrotationsmall[orange]{0.0}{0.2} 
    \end{tikzpicture}
  \end{center} 
\end{multicols}

\paragraph{Placinc a letter on the right}

\begin{multicols}{2}
 \begin{Verbatim}
  \begin{center}
    \begin{tikzpicture}[y=-1cm]
      \tikzrotationsmall[black][a]{0.0}{0.2} 
    \end{tikzpicture}
  \end{center} 
 \end{Verbatim}
 \columnbreak
  \begin{center}
    \begin{tikzpicture}[y=-1cm]
      \tikzrotationsmall[black][a]{0.0}{0.2} 
    \end{tikzpicture}
  \end{center} 
\end{multicols}

\paragraph{Placing a letter below}

\begin{multicols}{2}
 \begin{Verbatim}
  \begin{center}
    \begin{tikzpicture}[y=-1cm]
      \tikzrotationsmall[black][][b]{0.0}{0.2} 
    \end{tikzpicture}
  \end{center} 
 \end{Verbatim}
 \columnbreak
  \begin{center}
    \begin{tikzpicture}[y=-1cm]
      \tikzrotationsmall[black][][b]{0.0}{0.2} 
    \end{tikzpicture}
  \end{center} 
\end{multicols}

\paragraph{Placing a letter above}

\begin{multicols}{2}
  \begin{Verbatim}
    \begin{center}
      \begin{tikzpicture}[y=-1cm]
        \tikzrotationsmall[black][][][c]{0.0}{0.2} 
      \end{tikzpicture}
    \end{center} 
  \end{Verbatim}
  \columnbreak
  \begin{center}
    \begin{tikzpicture}[y=-1cm]
      \tikzrotationsmall[black][][][c]{0.0}{0.2} 
    \end{tikzpicture}
  \end{center} 
\end{multicols}


\paragraph{Example}
The QR decomposition of an upper Hessenberg matrix.

\begin{multicols}{2}
  \begin{Verbatim}
  \begin{center}
    \begin{tikzpicture}[y=-1cm]
      \foreach \j in {0,...,7} { 
        \tikzrotationsmall{\j/5}{\j/5}
      }
      \foreach \j in {0,...,8} { 
        \foreach \i in {\j,...,8} {
          \node at (\i/5+0.6,\j/5)
          [align=center,scale=0.6] {$\times$};
        } 
      }; 
    \end{tikzpicture}
  \end{center} 
  \end{Verbatim}
  \columnbreak
  \begin{center}
    \begin{tikzpicture}[y=-1cm]
      \foreach \j in {0,...,7} { 
        \tikzrotationsmall{\j/5}{\j/5}
      }
      \foreach \j in {0,...,8} { 
        \foreach \i in {\j,...,8} {
          \node at (\i/5+0.6,\j/5)
          [align=center,scale=0.6] {$\times$};
        } 
      }; 
    \end{tikzpicture}
  \end{center} 
\end{multicols}  

\subsection{uppertriangular}
Sometimes we depict an upper triangular matrix by an upper triangular
shape. Therefore we use the command \texttt{uppertriangular}

\begin{multicols}{2}
  \begin{Verbatim}
  \begin{center}
    \begin{tikzpicture}[y=-1cm]
      \uppertriangular{0.0}{0.0}
    \end{tikzpicture}
  \end{center} 
  \end{Verbatim}
  \columnbreak
  \begin{center}
    \begin{tikzpicture}[y=-1cm]
      \uppertriangular{0.0}{0.0}
    \end{tikzpicture}
  \end{center} 
\end{multicols}  

The upper triangle has a height and a width of 8 rotations or 9 entries. The
first argument is the x-coordinate of the upper left corner, the second argument
is the y-coordinate.

There are two additional optional argument. The first one changes the color.
\begin{multicols}{2}
  \begin{Verbatim}
  \begin{center}
    \begin{tikzpicture}[y=-1cm]
      \uppertriangular[green]{0.0}{0.0}
    \end{tikzpicture}
  \end{center} 
  \end{Verbatim}
  \columnbreak
  \begin{center}
    \begin{tikzpicture}[y=-1cm]
      \uppertriangular[green]{0.0}{0.0}
    \end{tikzpicture}
  \end{center} 
\end{multicols}  

The second one specifies the size in rotations.
\begin{multicols}{2}
  \begin{Verbatim}
  \begin{center}
    \begin{tikzpicture}[y=-1cm]
      \uppertriangular[black][5]{0.0}{0.0}
    \end{tikzpicture}
  \end{center} 
  \end{Verbatim}
  \columnbreak
  \begin{center}
    \begin{tikzpicture}[y=-1cm]
      \uppertriangular[black][5]{0.0}{0.0}
    \end{tikzpicture}
  \end{center} 
\end{multicols}  




\paragraph{Example}
The QR decomposition of an upper Hessenberg matrix.

\begin{multicols}{2}
  \begin{Verbatim}
  \begin{center}
    \begin{tikzpicture}[scale=1.66,y=-1cm]
      \foreach \j in {0,...,7} { 
        \tikzrotation{\j/5}{\j/5}
      }
      \uppertriangular{0.6}{0.0}

      \foreach \j in {0,...,4} { 
        \tikzrotation{\j/5}{\j/5+2.0}
      }
      \uppertriangular[black][5]{0.6}{2.0}
    \end{tikzpicture}
  \end{center} 
  \end{Verbatim}
  \columnbreak
  \begin{center}
    \begin{tikzpicture}[scale=1.66,y=-1cm]
      \foreach \j in {0,...,7} { 
        \tikzrotation{\j/5}{\j/5}
      }
      \uppertriangular{0.6}{0.0}

      \foreach \j in {0,...,4} { 
        \tikzrotation{\j/5}{\j/5+2.0}
      }
      \uppertriangular[black][5]{0.6}{2.0}
    \end{tikzpicture}
  \end{center} 
\end{multicols}  


\subsection{shiftthroughlr and shiftthroughrl}
We sometimes draw an arrow to indicate that a rotation was passed through an
upper triangular. We use \texttt{shiftthroughlr} for shifting a rotation from
left to right and \texttt{shiftthroughlr} for right to left.

\begin{multicols}{2}
  \begin{Verbatim}
  \begin{center}
    \begin{tikzpicture}[y=-1cm]
      \shiftthroughlr{0.0}{1.0}{0.0}
      \shiftthroughrl{2.0}{1.0}{0.5}
    \end{tikzpicture}
  \end{center} 
  \end{Verbatim}
  \columnbreak
  \begin{center}
    \begin{tikzpicture}[y=-1cm]
      \shiftthroughlr{0.0}{1.0}{0.0}
      \shiftthroughrl{2.0}{1.0}{0.5}
    \end{tikzpicture}
  \end{center} 
\end{multicols}
The first argument is the x-coordinate of the rotation before the shift through
operation, the second argument is the x-coordinate of the rotation after the
shift through operation, and the third argument is the y-coordinate.

There is an optional argument to change the color.

\begin{multicols}{2}
  \begin{Verbatim}
  \begin{center}
    \begin{tikzpicture}[y=-1cm]
      \shiftthroughlr[blue]{0.0}{1.0}{0.0}
      \shiftthroughrl[red]{2.0}{1.0}{0.5}
    \end{tikzpicture}
  \end{center} 
  \end{Verbatim}
  \columnbreak
  \begin{center}
    \begin{tikzpicture}[y=-1cm]
      \shiftthroughlr[blue]{0.0}{1.0}{0.0}
      \shiftthroughrl[red]{2.0}{1.0}{0.5}
    \end{tikzpicture}
  \end{center} 
\end{multicols}


\newpage{}
\paragraph{Example}
An example with rotations and upper triangular matrix looks like:

\begin{multicols}{2}
  \begin{Verbatim}
  \begin{center}
    \begin{tikzpicture}
      [scale=1.66,y=-1cm]
      \tikzrotation{0.0}{0.2}
      \tikzrotation{2.4}{0.2}
      \shiftthroughlr{0.0}{2.4}{0.2}
      \tikzrotation{2.2}{0.8}
      \tikzrotation{0.4}{0.8}
      \shiftthroughrl{2.2}{0.4}{0.8}
    \end{tikzpicture}
  \end{center} 
  \end{Verbatim}
  \columnbreak
  \begin{center}
    \begin{tikzpicture}
      [scale=1.66,y=-1cm]
      \tikzrotation{0.0}{0.2}
      \tikzrotation{2.4}{0.2}
      \shiftthroughlr{0.0}{2.4}{0.2}
      \tikzrotation{2.2}{0.8}
      \tikzrotation{0.4}{0.8}
      \shiftthroughrl{2.2}{0.4}{0.8}
    \end{tikzpicture}
  \end{center} 
\end{multicols}  


\subsection{similaritylr and similarityrl}
A similarity transformation can be used to bring a rotation from one side of a
matrix to the other. Therefore a curly arrow is used.

\begin{multicols}{2}
  \begin{Verbatim}
  \begin{center}
    \begin{tikzpicture}[scale=1.66,y=-1cm]
      \similaritylr{0.0}{1.0}{0.0}
      \similarityrl{2.0}{1.0}{0.5}
    \end{tikzpicture}
  \end{center} 
  \end{Verbatim}
  \columnbreak
  \begin{center}
    \begin{tikzpicture}[scale=1.66,y=-1cm]
      \similaritylr{0.0}{1.0}{0.0}
      \similarityrl{2.0}{1.0}{0.5}
    \end{tikzpicture}
  \end{center} 
\end{multicols}
The first argument is the x-coordinate of the rotation before the similarity
transformation, the second argument is the x-coordinate of the rotation after
the similarity transformation, and the third argument is the y-coordinate.

There is an optional argument to change the color.

\begin{multicols}{2}
  \begin{Verbatim}
  \begin{center}
    \begin{tikzpicture}[scale=1.66,y=-1cm]
      \similaritylr[blue]{0.0}{1.0}{0.0}
      \similarityrl[red]{2.0}{1.0}{0.5}
    \end{tikzpicture}
  \end{center} 
  \end{Verbatim}
  \columnbreak
  \begin{center}
    \begin{tikzpicture}[scale=1.66,y=-1cm]
      \similaritylr[blue]{0.0}{1.0}{0.0}
      \similarityrl[red]{2.0}{1.0}{0.5}
    \end{tikzpicture}
  \end{center} 
\end{multicols}


\paragraph{Example}
An example with rotations and an upper triangular matrix looks like:

\begin{multicols}{2}
  \begin{Verbatim}
  \begin{center}
    \begin{tikzpicture}
      [scale=1.66,y=-1cm]
      \tikzrotation{0.0}{0.2}
      \tikzrotation{2.4}{0.2}
      \similaritylr{0.0}{2.4}{0.2}
      \tikzrotation{2.2}{0.8}
      \tikzrotation{0.4}{0.8}
      \similarityrl{2.2}{0.4}{0.8}
    \end{tikzpicture}
  \end{center} 
  \end{Verbatim}
  \columnbreak
  \begin{center}
    \begin{tikzpicture}
      [scale=1.66,y=-1cm]
      \tikzrotation{0.0}{0.2}
      \tikzrotation{2.4}{0.2}
      \similaritylr{0.0}{2.4}{0.2}
      \tikzrotation{2.2}{0.8}
      \tikzrotation{0.4}{0.8}
      \similarityrl{2.2}{0.4}{0.8}
    \end{tikzpicture}
  \end{center} 
\end{multicols}  


\subsection{turnoverrl and turnoverlr}
With a turnover a rotation is moved through an ascending or descending sequence
of rotations. This can be done from left to right and from right to left. We
depict it sometimes with an arrow.


\begin{multicols}{2}
  \begin{Verbatim}
  \begin{center}
    \begin{tikzpicture}[scale=1.66,y=-1cm]
      \turnoverlr{-0.2}{0.2}{0.4}{0.4}
      \turnoverrl{0.4}{0.6}{-0.2}{0.8}
    \end{tikzpicture}
  \end{center} 
  \end{Verbatim}
  \columnbreak
  \begin{center}
    \begin{tikzpicture}[scale=1.66,y=-1cm]
    \turnoverlr{-0.2}{0.2}{0.4}{0.4}
    \turnoverrl{0.4}{0.6}{-0.2}{0.8}
    \end{tikzpicture}
  \end{center} 
\end{multicols}  

The first argument is the x-coordinate of the rotation before the turnover, the
second argument is the y-coordinate, the third and fourth arguments are the x-
and y-coordinate after the turnover. An optional argument changes the color.

\paragraph{Example}
\begin{multicols}{2}
  \begin{Verbatim}
  \begin{center}
    \begin{tikzpicture}[scale=1.66,y=-1cm]
      \tikzrotation{0.2}{0.2}
      \tikzrotation{0.0}{0.4}
      \tikzrotation{-0.2}{0.6}
      \tikzrotation{0.0}{0.8}
      \tikzrotation{0.2}{1.0}
      \tikzrotation[blue]{-0.2}{0.2}
      \turnoverlr[blue]{-0.2}{0.2}{0.4}{0.4}
      \tikzrotation[blue]{0.4}{0.4}
      \tikzrotation[orange]{0.4}{0.8}
      \turnoverrl[orange]{0.4}{0.8}{-0.2}{1.0}
      \tikzrotation[orange]{-0.2}{1.0}
    \end{tikzpicture}
  \end{center} 
  \end{Verbatim}
  \columnbreak
  \begin{center}
    \begin{tikzpicture}[scale=1.66,y=-1cm]
    \tikzrotation{0.2}{0.2}
    \tikzrotation{0.0}{0.4}
    \tikzrotation{-0.2}{0.6}
    \tikzrotation{0.0}{0.8}
    \tikzrotation{0.2}{1.0}
    \tikzrotation[blue]{-0.2}{0.2}
    \turnoverlr[blue]{-0.2}{0.2}{0.4}{0.4}
    \tikzrotation[blue]{0.4}{0.4}
    \tikzrotation[orange]{0.4}{0.8}
    \turnoverrl[orange]{0.4}{0.8}{-0.2}{1.0}
    \tikzrotation[orange]{-0.2}{1.0}
    \end{tikzpicture}
  \end{center} 
\end{multicols}  


\subsection{transferbulgelr and transferbulgerl}
Similar to the similarity transformation one can move a rotation from one matrix
of a matrix pencil to the other. We depict this also with a curly arrow.

\begin{multicols}{2}
  \begin{Verbatim}
  \begin{center}
    \begin{tikzpicture}[y=-1cm]
    \transferbulgelr{0.2}{2.2}{0.0}
    \transferbulgerl{2.6}{0.6}{0.8}
    \end{tikzpicture}
  \end{center} 
  \end{Verbatim}
  \columnbreak
  \begin{center}
    \begin{tikzpicture}[y=-1cm]
    \transferbulgelr{0.2}{2.2}{0.0}
    \transferbulgerl{2.6}{0.6}{0.8}
    \end{tikzpicture}
  \end{center} 
\end{multicols}  

The first argument is the x-coordinate of the rotation before the equivalence
transformation, the second argument is the x-coordinate of the rotation after,
and the third argument is the y-coordinate.

There is an optional argument to change the color.

\begin{multicols}{2}
  \begin{Verbatim}
  \begin{center}
    \begin{tikzpicture}[y=-1cm]
    \transferbulgelr[orange]{0.2}{2.2}{0.0}
    \transferbulgerl[gray]{2.6}{0.6}{0.8}
    \end{tikzpicture}
  \end{center} 
  \end{Verbatim}
  \columnbreak
  \begin{center}
    \begin{tikzpicture}[y=-1cm]
    \transferbulgelr[orange]{0.2}{2.2}{0.0}
    \transferbulgerl[gray]{2.6}{0.6}{0.8}
    \end{tikzpicture}
  \end{center} 
\end{multicols}  


\paragraph{Example}
%\begin{multicols}{2}
  \begin{Verbatim}
  \begin{center}
    \begin{tikzpicture}[scale=1.66,y=-1cm]
    \uppertriangular{0.0}{0.0}
    \node[above] at (2.2,1.2) {,};      
    \uppertriangular{2.7}{0.0}
    \tikzrotation{4.5}{0.4}
    \tikzrotation{1.8}{0.4}
    \transferbulgerl{4.5}{1.8}{0.4}
    \end{tikzpicture}
  \end{center} 
  \end{Verbatim}
  %\columnbreak
  \begin{center}
    \begin{tikzpicture}[scale=1.66,y=-1cm]
    \uppertriangular{0.0}{0.0}
    \node[above] at (2.0,1.2) {,};      
    \uppertriangular{2.2}{0.0}
    \tikzrotation{4.0}{0.4}
    \tikzrotation{1.8}{0.4}
    \transferbulgerl{4.0}{1.8}{0.4}
    \end{tikzpicture}
  \end{center} 
%\end{multicols}  

\subsection{drawbrace}

The command drawbrace can be used to provide further information for a sequence
of rotations.

\begin{multicols}{2}
  \begin{Verbatim}
  \begin{center}
    \begin{tikzpicture}[y=-1cm]
      \drawbrace{0.0}{1.0}{0.0}{$=C_{1}\dotsc C_{n-1}$}
    \end{tikzpicture}
  \end{center} 
  \end{Verbatim}
  \columnbreak
  \begin{center}
    \begin{tikzpicture}[y=-1cm]
      \drawbrace{0.0}{1.0}{0.0}{$=C_{1}\dotsc C_{n-1}$}
    \end{tikzpicture}
  \end{center} 
\end{multicols}  

The first argument is the x-coordinate of the leftmost rotation, the second
argument the x-coordinate of the rightmost rotation. The third argument is the
y-coordinate of the brace and the last argument is the text below. There is an
optional argument for changing the color.

\begin{multicols}{2}
  \begin{Verbatim}
  \begin{center}
    \begin{tikzpicture}[y=-1cm]
      \drawbrace[blue]{0.0}{1.0}{0.0}{$=C_{1}\dotsc C_{n-1}$}
    \end{tikzpicture}
  \end{center} 
  \end{Verbatim}
  \columnbreak
  \begin{center}
    \begin{tikzpicture}[y=-1cm]
      \drawbrace[blue]{0.0}{1.0}{0.0}{$C_{1}\dotsc C_{n-1}$}
    \end{tikzpicture}
  \end{center} 
\end{multicols}  

\newpage{}
\paragraph{Example}
\begin{multicols}{2}
  \begin{Verbatim}
  \begin{center}
    \begin{tikzpicture}[scale=1.66,y=-1cm]
      \foreach \j in {0,...,7} { 
        \tikzrotation{\j/5}{\j/5}
      }
      \drawbrace{0.0}{1.4}{2.0}{$G_{1}
        \dotsc G_{n-1}$}
    \end{tikzpicture}
  \end{center} 
  \end{Verbatim}
  \columnbreak
  \begin{center}
    \begin{tikzpicture}[scale=1.66,y=-1cm]
      \foreach \j in {0,...,7} { 
        \tikzrotation{\j/5}{\j/5}
      }
      \drawbrace{0.0}{1.4}{2.0}{$G_{1}
        \dotsc G_{n-1}$}
    \end{tikzpicture}
  \end{center} 
\end{multicols}  


\section{New in version 1.1}

All arrows (shiftthroughlr and shiftthroughrl, similaritylr and similarityrl,
turnoverrl and turnoverlr, and transferbulgelr and transferbulgerl) do now have
a second optional argument which allows to place a short text above the arrow.

\begin{multicols}{2}
  \begin{Verbatim}
  \begin{center}
    \begin{tikzpicture}[scale=1.66,y=-1cm]
      \tikzrotation{0.0}{0.0}%
      \tikzrotation{0.2}{0.2}%   
      \node at (0.6,0.0){$\times$};%
      \node at (0.8,0.2){$\times$};%
      \node at (1.0,0.4){$\times$};%
      \tikzrotation[red]{0.6}{0.0}
      \shiftthroughrl[red][(1)]{1.6}{0.4}{0.0}
      \tikzrotation[red]{0.4}{0.0}
      \turnoverrl[red][(2)]{0.4}{0.0}{-0.2}{0.2}
      \tikzrotation[red]{-0.2}{0.2}    
      \similaritylr[red][(3)]{-0.2}{0.0}{0.2}
      \tikzrotation[red]{2.0}{0.2}
      \shiftthroughrl[red][(4)]{2.0}{0.2}{0.2}
    \end{tikzpicture}
  \end{center} 
  \end{Verbatim}
  \columnbreak
  \begin{center}
    \begin{tikzpicture}[scale=1.66,y=-1cm]
      \tikzrotation{0.0}{0.0}%
      \tikzrotation{0.2}{0.2}%   
      \node at (0.6,0.0){$\times$};%
      \node at (0.8,0.2){$\times$};%
      \node at (1.0,0.4){$\times$};%
      \tikzrotation[red]{1.6}{0.0}
      \shiftthroughrl[red][(1)]{1.6}{0.4}{0.0}
      \tikzrotation[red]{0.4}{0.0}
      \turnoverrl[red][(2)]{0.4}{0.0}{-0.2}{0.2}
      \tikzrotation[red]{-0.2}{0.2}    
      \similaritylr[red][(3)]{-0.2}{2.0}{0.2}
      \tikzrotation[red]{2.0}{0.2}
      \shiftthroughrl[red][(4)]{2.0}{0.2}{0.2}
    \end{tikzpicture}
  \end{center} 
\end{multicols}  



\end{document}

%\begin{multicols}{2}
  % \begin{Verbatim}
  % \begin{center}
  %   \begin{tikzpicture}[y=-1cm]
  %   \end{tikzpicture}
  % \end{center} 
  % \end{Verbatim}
%  \columnbreak
  % \begin{center}
  %   \begin{tikzpicture}[y=-1cm]
  %   \end{tikzpicture}
  % \end{center} 
%\end{multicols}  


%%% Local Variables: 
%%% mode: LaTeX
%%% TeX-PDF-mode:t
%%% TeX-engine: luatex
%%% auto-fill-function:nil
%%% mode:auto-fill
%%% flyspell-mode:nil
%%% mode:flyspell
%%% ispell-local-dictionary: "american"
%%% End: 
